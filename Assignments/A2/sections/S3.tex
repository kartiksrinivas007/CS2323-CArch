\section{Section - 3}
\vspace{5pt}
\hrule
\vspace{5pt}
The first instruction will load bits 12 through 31 inside the register x3, using the
value that is stored ,
the next instruction will load half-word (in an unsigned manner)
\\
Half word  = 16 bits  = 4 hexadecimal digits.
The lower 4 Hexadecimal digits from byte 0 and byte 1 will be used for storage within the least significant bits
of the register x3.
\\
$\textbf{x3 = 0x000000000000a5a5} $
\\
Now if instead we load in a signed manner, the CPU will first check the MSB 
of the 16 bit number being loaded. This a5a5 has the sign bit of $\textbf{0xa = 1010}$
.Therefore there is an \textbf{extension} of 1's while loading (which is equivalent to the 
$\textbf{x3} = \textbf{0x}ffffffffffff\textbf{a5a5}$
\\
Now loading from  an offset of 2 will load \textbf{Byte 3 and Byte 4} into the register along with thw usual sign extension
\\
$\textbf{x3} = \textbf{0x}ffffffffffff\textbf{$a5_{byte-4}a5_{byte-3}$}$
\\
Now we load the entire double from Byte-0 onwards, we will get the first double word inside
the register.
\\
$\textbf{x3} = \textbf{0x39933939a55aa5a5}$
\\
Now we load from \textbf{Byte 5 onwards}. Since the memory is contiguous, it will start reading into the \textbf{next double word}
The number stored is then a mix of the higher bytes from the first doubble word and the lower of the second one.
$\textbf{x3} = \textbf{0x$39933939_{second} 39933939_{first}$}$
\\
Next instruction will just load the eight'th byte(7 have been offset)
\\
$\textbf{x3} = \textbf{0x0000000000000039}$
\\
Now even if I use \textbf{lb} , it won't make a difference because \textbf{0x3 = 0011} and the sign extension 
will be just of 0's.\\
$\textbf{x3} = \textbf{0x0000000000000039}$
\\
Now , 7'th Byte actually is \textbf{0x93 = 1001 0011}, thsi would imply a sign extension when loaded
\\
$\textbf{x3} = \textbf{0xffffffffffffff93}$
\\
For code please see \cite{CS2323}
\vspace{5pt}
\hrule
