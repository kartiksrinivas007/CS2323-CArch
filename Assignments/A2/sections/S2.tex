
\section{Section 2}
\vspace{7pt}
\hrule
\vspace{7pt}
The load immediate instruction can be considered as a "pseudo instruction".
\begin{enumerate}
    \item li t0 -1  = addi x5 x0 -1
    \\Clearly these both have the same meaning, we load into the register t0  = x5 after adding a '-1' to x0 = 0
    \item li t0 0xFFFFFFFF
    \\ Note that the hexadecimal 0xFFFF FFFF is equivalent to -1  ( it is the 2's complement of 1, so both the instructions are the same)
    \item li t0 223
    \\ 223 is only 8 bits long, so it within the 12 bit immediate that addi can handle, 
    so the instruction addi x5 x0 223, remains valid
    \item li t0 1234 
     \\ Note that 1234 is 11 bits long which is still within the valid length of 12 bits for immediates , so addi x5 x0 1234 
     will still be correct $(1234 < 2^{11} - 1)$
    \item li t0 0x123456000
    \\ Now there has been a change, there are 9 hexadecimal digits which equals 36 normal bits, \textbf{too large} for an immediate, so it needs to be added by breaking it into pieces.
    The first step done by the simulator is to load 0x00092 into bits 31 : 12 using lui, then it adds the number -1493 = 0xffffffffa2b\\
    on addition , the number becomes 0x00000000091a2b, this number on shifitng left by 13 will then become 0x123456000.
\end{enumerate}
\vspace{7pt}
\hrule