\section{Analysis of Reads}
\subsubsection{Program P1}
Say that we are analyzing the case with four blocks and direct mapped (non assoscative case)
P1 is a program where the register \verb |x12| is read and is incremented by 8 every single time, 
the main idea is to see \textcolor{blue}{once every four times} there is a cache miss. Therefore the average cache perofmance 
should converge to $3/4 \approx 0.75$. \textbf{This won't change if we change the number of Line's in the cache}, simply because our memory
access pattern is \textcolor{red}{Sequential} in nature, i.e. the next memory location accessed for read is always \textbf{new}.
There are no instances where I access a previously accessed memory location \textcolor{red}{with a different Tag} The only way the performance of the 
cache will change is if we change the number of blocks that are addressed by a particular Tag will change. i.e increase in blocks as illustrated by the following Table
\\
\begin{center}
\begin{tabular}{c | c | c | c | c}
L1 & Blocks & Lines & Ways &  Hit rate\\
\hline
8,8 & 2 & 32 & 1 & 0.5\\
\hline
8,8 & 2 & 16 & 1 & 0.5\\
\hline
8,8 & 2 & 8 & 1 & 0.5\\
\hline
8,8 & 4 & 32 & 1 & 0.7424\\
\hline
8,8 & 8 & 32 & 1 & 0.8636\\
\hline
8,8 & 16 & 32 & 1 & 0.9242\\
\hline
8,16 &  2 & 32 & 1 & 0.5\\
\hline
8,16 & 4 & 32 & 1 & 0.7424\\
\hline
8,16 & 8 & 32 & 1 & 0.8636\\
\hline
8,16 & 16 & 32 & 1 & 0.9242\\
\hline
\end{tabular}
\end{center}
\subsubsection{Program P2}
The main difference as compared to P1 is that \verb|x12| is incremented by \verb|x15| at every iteration, and it is \textbf{reset to the value of the register} \verb|x15|
which falls back to previously accessed memory locations. So, \textcolor{red}{larger line sizes(up to some extent)} will help use imporve the hit rate. The number or cache acesses 
will be the same at the value of 64. But the hit rate will be higher(because the cache conflicts will decrease) If I increase the number of ways then the performance will increase(when you have a smaller number
of lines), again, as the cache will start having more space and the conflicts will decrease. Usually in terms of the cycles of the outer loop of the program, we get
Loops 1, 3 Have cache misses and Loops 2, 4, 5, 6, 7, 8 have cache hits, so the performance of the cache will converge to $6/8 \approx 3/4 \approx 0.75$ again(in the case of four blocks)
\textbf{Increasing the number of blocks again improves the performance}. Excessive lines (after a certain point) will not help us improve the performance, because the cache will not be utilized that much anyway!
\\
\begin{center}
\begin{tabular}{c | c | c | c | c}
L1 & Blocks & Lines & Ways &  Hit rate\\
\hline
8,8 & 32 & 8 & 1 & \textcolor{blue}{0.9545}\\
\hline
8,8 & 16 & 8 & 1 & 0.9242\\
\hline
8,8 & 8 & 8 & 1 & 0.8636\\
\hline
8,8 & 4 & 8 & 1 & \textcolor{red}{0.0454}   \\
\hline
8,8 & 32 & 32 & 1 & \textcolor{blue}{0.9545}\\
\hline
\end{tabular}
\end{center}

\section{Analysis of Write's}

\subsubsection{Write-Back}
In this section of the code we can see the \textcolor{red}{Dirty} bits etc. During \textbf{\textcolor{red}{Write Allocate}} the memory access patters will 
remain the same!, because WA behaves just like read miss when the block is brough to the cache. So the hit rates will remain the same in Write allocate scenario.
The only difference is that the \textcolor{red}{Dirty} bit will be set to 1. Writebacks only occur when there is an \textbf{evicition} from the cache.
\subsection{Problem -1}
In this problem the memory acesses are very sequential in nature and therefore unless the size of the cache is very small there will not be many write-backs
\subsection{Problem -2}
In this case, if you are given enough blocks or lines, the number of Writebacks will be 0!. This is because the memory access pattern of the second program conforms to that of 
hits on prevously accessed locations or just new locations (which are not in the cache) , very rarley do we see  a case where the tag matches but the index does not.
\\
\begin{center}
\begin{tabular}{c | c | c | c}
L1 & Blocks & Lines & Writebacks\\
\hline
8,8 & 2 & 32 & 0\\
\hline
8,8 & 32 & 2 & 0\\
\hline
16, 16 & 32 & 2 & \textcolor{red}{126}\\
\hline
8, 8 & 2 & 8 & \textcolor{red}{248}\\
\hline
\end{tabular}
\end{center}
The red numbers show that if the space in the cache is too small  or if the number of accesses is too large, then the number of conflicts  = Writebacks are quite large and the performance is poor.

