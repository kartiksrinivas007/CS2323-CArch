
\section{Problem 1}
The total time is added for both the read and write onto any intermediate register is assumed to be 12ps.
Then the time for each of the stages will \textcolor{blue}{increase by 12ps}.
This means that the cycle time($T_c$) =$\max(T_i) + T_{reg} = 145 + 12 = 157ps.$

\subsection{Part a}

$$
        1000 \times T_c \ s = 157 \times 10^{-9} \ s = 157ns 
$$

\subsection{Part b}
A small assumption, since 1000 instructions is large, the time for the setup region of the pipelining process has been ignored.
In case there are hazards in 20 \% of the instructions which incur a one cycle stall.
We can see this in two ways, either the CPI increases to 1 + 0.2 or we can calculate the number of cycles as follows:
$$
 T_c \times 1000 \times  cpi  = 157 * ( 1 + 0.2 * 1 ) = 188.4 \ ns
$$
The second way is to calculate the number of cycles.
$$
 1000 * 0.2 * 1  + 1000 = 1200  \ \text{cycles}, 
 T = 1200 * 157 \ ps = 188.4 \ ns
$$

\section{Problem 2}
The time must be sufficient for the longest delay in the pipelined system, as the ALU latency increases by $250 \ ps$
the answer will be 
$$
800/5  + 250 = 250 + 160 \ ps = 410 \ ps
$$
The processor P1 and P2 will excute $x$ and $9x/10$ number of instructions.
P1 would take 160 ps per cycle of instruction and P2 would take 410 ps , therefore
$$
T_{p2} = 410 \times  9x/10  > 160 \times x =  T_{p1}
$$ 
So \textcolor{blue}{P2 would be slower than P1}

If a piece of code has 5000 instructions on P1 , then the number of instructions on P2 = 4500
so the execution times are 
\\
$$
T_{p1} = 160 \times 5000 \ ps = 80 e4 \ ps 
$$

$$
T_{p2} = 410 \times 4500 \ ps = 184.5 e4 \ ps
$$
\section{Problem 3}
The code after inserting nops is as follows, The \textbf{ add instruction will add two nops and load use will add 3 nops.}
However the load instruction has a conflict on the next to next instruction, so only 2 nops are required for the load one too.
The total number of Nops = 6.
\subsection{Without Forwarding}
\begin{lstlisting}
add x14 x12 x11
addi x0 x0 0 # nop
addi x0 x0 0 # nop
add x15 x14 x12
ld  x13 8(x13)
ld  x12  0(x14) # fine, no update 2 steps above
addi x0 x0 0 #nop because x13 has been updated above
addi x0 x0 0 #nop because x13 has been updated above
and x13 x15 x13 
addi x0 x0 0 # nop
addi x0 x0 0 # nop
ld x11 4(x13)
sd x13 0(x15)
\end{lstlisting}


\subsection{With Forwarding}
In case I use forwarding then the  number of nops required for simple data hazards is zero , while as  a load use hazard will take a single
nop, ( btu that is already covered under an existing instruction)
\begin{lstlisting}
add x14 x14 x11
add x15 x14 x12 
ld x13 8(x13)
ld x12 0(x14) # effectively behaves like a nop to x13
and x13, x15, x13
ld x11, 4(x13) # no nop
sd x13, 0(x15)
\end{lstlisting}
So \textbf{NO Nops} will be required with forwarding in this case.
